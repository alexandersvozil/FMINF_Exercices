\documentclass [11pt]{article}
\usepackage{latexsym}
\usepackage{amssymb}
\usepackage{epsfig} 
\usepackage{enumerate}
\usepackage{xspace}
\usepackage{amsmath}
\usepackage{color}
\usepackage{xcolor}
\usepackage{listings}

\usepackage{caption}
\DeclareCaptionFont{white}{\color{white}}
\DeclareCaptionFormat{listing}{\colorbox{gray}{\parbox{\textwidth}{#1#2#3}}}
\captionsetup[lstlisting]{format=listing,labelfont=white,textfont=white}




   \textwidth      15cm
   \textheight     23cm
   \oddsidemargin 0.5cm
   \topmargin    -0.5cm
   \evensidemargin\oddsidemargin

 \newcommand{\nop}[1]{}


   \pagestyle{plain}
   \bibliographystyle{plain}


\title{Formale Methoden der Informatik \\
Block 1: Computability and Complexity }
\author{Exercises 1-10 \\
Submission deadline: 21.04.2014}
\date{SS 2014}


  \newtheorem{theorem}{Theorem}
  \newtheorem{lemma}[theorem]{Lemma}
  \newtheorem{corollary}[theorem]{Corollary}
  \newtheorem{proposition}[theorem]{Proposition}
  \newtheorem{conjecture}[theorem]{Conjecture}
  \newtheorem{definition}[theorem]{Definition}
  \newtheorem{example}[theorem]{Example}
  \newtheorem{remark}[theorem]{Remark}
  \newtheorem{exercise}[theorem]{Exercise}

  \newcommand{\ra}{\rightarrow}
  \newcommand{\Ra}{\Rightarrow}
  \newcommand{\La}{\Leftarrow}
  \newcommand{\la}{\leftarrow}
  \newcommand{\LR}{\Leftrightarrow}

  \renewcommand{\phi}{\varphi}
  \renewcommand{\theta}{\vartheta}


\newcommand{\ccfont}[1]{\protect\mathsf{#1}}
\newcommand{\NP}{\ccfont{NP}}

\newcommand{\NN}{\textbf{N}}
\newcommand{\IN}{\textbf{Z}}
\newcommand{\bigO}{\mathrm{O}}
\newcommand{\bigOmega}{\Omega}
\newcommand{\bigTheta}{\Theta}
\newcommand{\REACHABILITY}{\mbox{\bf REACHABILITY}}
\newcommand{\MAXFLOW}{\mbox{\bf MAX FLOW}}
\newcommand{\MAXFLOWD}{\mbox{\bf MAX FLOW(D)}}
\newcommand{\MAXFLOWSUB}{\mbox{\bf MAX FLOW SUBOPTIMAL}}
\newcommand{\MATCHING}{\mbox{\bf BIPARTITE MATCHING}}
\newcommand{\TSP}{\mbox{\bf TSP}}
\newcommand{\TSPD}{\mbox{\bf TSP(D)}}

\newcommand{\ThreeCol}{\mbox{\bf 3-COLORABILITY}}
\newcommand{\TwoCol}{\mbox{\bf 2-COLORABILITY}}
\newcommand{\kCol}{\mbox{\bf k-COLORABILITY}}
\newcommand{\HamPath}{\mbox{\bf HAMILTON-PATH}}
\newcommand{\HamCycle}{\mbox{\bf HAMILTON-CYCLE}}

\newcommand{\ONESAT}{\mbox{\bf 1-IN-3-SAT}}
\newcommand{\MONONESAT}{\mbox{\bf MONOTONE 1-IN-3-SAT}}
\newcommand{\kSAT}{\mbox{\bf k-SAT}}
\newcommand{\NAESAT}{\mbox{\bf NAESAT}}
\newcommand{\CLIQUE}{\textbf{CLIQUE}\xspace} 
\newcommand{\VC}{\textbf{VERTEX COVER}\xspace}

\newcommand{\VerCov}{\mbox{\bf Vertex Cover}\xspace}
\newcommand{\SetCov}{\mbox{\bf Set Cover}\xspace}
\newcommand{\EmplSched}{\mbox{\bf Employee Scheduling}\xspace}



\renewcommand{\labelenumi}{(\alph{enumi})}

%%% useful macros for Turing machines:
\newcommand{\blank}{\sqcup}
\newcommand{\ssym}{\triangleright}
\newcommand{\esym}{\triangleleft}
\newcommand{\halt}{\mbox{h}}
\newcommand{\yess}{\mbox{``yes''}}
\newcommand{\nos}{\mbox{``no''}}
\newcommand{\lmove}{\leftarrow}
\newcommand{\rmove}{\rightarrow}
\newcommand{\stay}{-}
\newcommand{\diverge}{\nearrow}
\newcommand{\yields}[1]{\stackrel{#1}{\rightarrow}}

\newcommand{\HALTING}{\mbox{\bf HALTING}}

\newcommand{\true}{{\it true}}
\newcommand{\false}{{\it false}}


%\newcommand{\samplesolution}[1]{\noindent {\bf Sample solution.}  #1}
\newcommand{\solution}[1]{\noindent {\bf Solution.}  #1}



%%%%%%%%%%%%%%%%%%%%%%%%%%%%%%%%%%%%%%%%%%%%%%%%%%%%%%%%%%%%%%%%%%%

\begin{document}

\maketitle


\begin{exercise}
  Prove that the following problem is undecidable:

  \begin{center}
    \fbox{
      \begin{minipage}[c]{.9\linewidth}
        \textbf{DOUBLE}

        \medskip INSTANCE: A pair $(\Pi,I)$, where $I$ is a string and $\Pi$
        is a program that takes one string as input and outputs a string.
        \\
        QUESTION: Does the program $\Pi$ on the string $I$ as input return as
        output the string $I\,{+}\,I$? Here $+$ is the operator for string
        concatenation.
      \end{minipage}
    }
  \end{center}
  Prove the undecidability by providing a reduction from the \textbf{HALTING}
  problem to \textbf{DOUBLE}, and arguing that your reduction is correct.
\end{exercise}


\solution{The proof proceeds by reduction from HALTING to DOUBLE.\\
Let$(\Pi,I)$ be an arbitrary instance of the \textbf{HALTING} problem, i.e., $\Pi$ is a program that takes one string and I is an Input for $\Pi$.
From this, we construct an instance $(\Pi',I)$ of \textbf{DOUBLE} by building $(\Pi')$ as follows:\\}

\begin{lstlisting}[label=haltingtodouble,caption=$\Pi'$,mathescape, numbers=left, numberstyle=\tiny]
String $\Pi'$ (String S) {
  call $\Pi$(S);
  return S+S;
}
\end{lstlisting}
It remains to show that the following equivalence holds:\\
$\Pi$ halts on I $\Leftrightarrow$ $\Pi'$ returns I+I\\
$\Rightarrow$\\
Assume $\Pi$ halts on I. This means line three is reached and I+I is returned.\\
$\Leftarrow$\\
Assume $\Pi'$ returns I+I. This implies, that the program must have passed line two.
This implies that $\Pi$ halts on I because it means running $\Pi$ on I and returning from it.
$\square$

\begin{exercise}
 Prove that \textbf{DOUBLE} is semi-decidable. To this end, provide a semi-decision procedure and justify your solution.
\end{exercise}


\solution{
We can write an interpreter program $\Pi_i$ such that:\\
\begin{itemize}
  \item{$\Pi_i$ takes as input a source code $\Pi$ and an Input I for $\Pi$}
  \item{$\Pi_i$ parses $\Pi$ and simulates a run of $\Pi$ on I.}
  \item{If the simulation of $\Pi$ on I reaches the end and $\Pi$ returns I+I, $\Pi_i$ returns true}
  \item{If the simulation of $\Pi$ on I reaches the end and $\Pi$ doesn't return I+I, $\Pi_i$ returns false}
  \item{If the simulation of $\Pi$ on I doesn't reach the end, then, clearly $\Pi_i$ cannot return any value }
\end{itemize}
}


\begin{exercise}
  Give a formal proof that \textbf{SUBSET SUM} is in $\NP$, i.e. define a
  certificate relation and discuss that it is polynomially balanced and
  polynomial-time decidable.
  
\medskip

\noindent \textbf{SUBSET SUM:} \\
INSTANCE: A finite set of integer numbers $S=\{a_1, a_2, \ldots, a_n\}$ and an integer number $t$. \\
QUESTION: Does there exist a subset $S'\subseteq S$, s.t.\ the sum of the elements in $S'$  is equal to $t$, i.e. $\big(\sum_{a_i \in S'} a_i\big) = t$?
  
\end{exercise}


\solution{
  As proof I provide a certificate relation R and discuss why it is polynomially balanced and polynomial-time decidable:\\

  $R=\{((S,t),S') \ | \  S' \subseteq S \wedge \big(\sum_{a_i \in S'} a_i\big) = t \}$\\
  \begin{itemize}
    \item{
        R is certificate relation by construction: (S,t) is a positive instance of \textbf{SUBSET SUM} $\leftrightarrow$
        There is a subset of S, $S'$ for which if all the integers in the set are added provide a sum which is equal to t. $\leftrightarrow$ 
      $((S,t),S')\in R$}
    \item{R is polynomially balanced because $S'$ is a subset of S.}
    \item{R is polynomial-time decidable because adding numbers in set $S'$ and then comparing the sum to d is in $\bigO(|S'|+1)$}.
  \end{itemize}
}


\begin{exercise}
  \label{ex:partition}
  Formally prove that \textbf{PARTITION} is $\NP$-complete. For this you may use
  the fact that \textbf{SUBSET SUM} is $\NP$-complete.   

\medskip     
   
\noindent \textbf{PARTITION:} \\
INSTANCE: A finite set of $n$ positive integers $P=\{p_1, p_2, \ldots, p_n\}$. \\
QUESTION: Can the set $P$ be partitioned into two subsets $P_1, P_2$ such that the sum of the numbers in $P_1$ equals the sum of the numbers in $P_2$? 
  

\end{exercise}



\solution{ 
  To formally prove that \textbf{PARTITION} is $\NP$-complete we need to first show that it is in NP.
  Therefore we need to provide a certificate relation R and discuss why it is polynomially balanced and polynomial-time decidable:\\
  $R=\{(P,P_1)) \ | \  P_1 \subseteq P \wedge \big(\sum_{a_i \in P_1} a_i\big) = \big(\sum_{a_i \in P \setminus P_1} a_i\big) \}$\\
  \begin{itemize}
    \item{
        R is certificate relation by construction: (P,$P_1$) is a positive instance of \textbf{PARTITION} $\leftrightarrow$
        There is a subset of P, $P_1$ for which if you partion P with $P_1$ you get a second subset $P_2$. If the integers in both sets are added up (seperately) they are equal. $\leftrightarrow$ 
      $((P,P_1)\in R$}
    \item{R is polynomially balanced because $P_1$ is a subset of P.}
    \item{R is polynomial-time decidable because adding $|R|$ integers and then comparing two numbers is in P}.
  \end{itemize}
 We proceed our proof by reducing \textbf{SUBSET SUM} to \textbf{PARTITION} in order to show hardness.
 Let an arbitrary instance of \textbf{SUBSET SUM} be given by a Set S and an integer t.\\
 Then we construct the following instance of \textbf{PARTITION}: \\
 b = $\big(\sum_{a_i \in S} a_i \big) $\\
 $P = S \cup b+t \cup 2b-t$ \\
 The reduction can obviously be done in polynomial time, because adding two elements to a set is in P.
 To prove the correctness of the reduction, we must show that (S,t) has a subset $S'$ where the sum of the elements equal to t $\leftrightarrow$ 
 the resulting instance P of \textbf{PARTITION} is positive.\\
 $\rightarrow$\\
 We need to show that under the assumption of a positive instance (S,t) of \textbf{SUBSET SUM} the resulting instance P of \textbf{PARTITION}is positive:\\
 Assume (S,t) has a subset $S'$ where the sum of the elements equal to t. By our reduction  $P = S \cup b+t \cup 2b-t$ \\
 Let $P_1$ = $S' \cup (2b-t)$ this implies that $P_2$ = $S \setminus S' \cup (b+t)$ by the definition of \textbf{PARTITION}. As t = $\big(\sum_{a_i \in S'} a_i \big) $ and $t+(2b-t)=2b$ the sum of the elements in 
 $P_1$ is equal to 2b. The sum of the elements in $S \setminus S'$ is equal to $b-t$ and  $(b-t) +(b+t) = 2b$. Thus the sum of the elements in 
 $P_1$ and $P_2$ are equal and we have a positive resulting instance P of \textbf{SUBSET SUM}.\\
 $\leftarrow$\\
 Suppose that P is a positive instance of \textbf{PARTITION}. Now we need to show, that (S,t) is a positive instance of \textbf{SUBSET SUM}.
 Because of the reduction $P = S \cup b+t \cup 2b-t$ and we can assume that there exist two sets $P_1$ and $P_2$ which partition $P$.
 The only way to partition P into two sets is to exactly split it into two sets where the sum of the elements add up to 2b, because the total sum of the elements in P is 4b.
 The only way to do this is to take some elements out of S which when summed up equal to the value t and add the element $(2b-t)$. (The other set is formed by $S \setminus S' \cup (b+t)$).
 Because we assumed that our set can be partioned we can be sure that there are elements of S which when summed up equal to the value t.
 Let the elements of S which when summed up equal to the value t form the Set $S'$. This is the definition of a positive instance of \textbf{SUBSET SUM}.
$\square$
}
\newpage

\begin{exercise}
    Provide a reduction from \textbf{VERTEX COVER} to \textbf{SET COVER}. Additionally, prove the ``$\Rightarrow$'' direction in the proof of correctness of the reduction,  i.e.  prove the following statement: if
    a \textbf{VERTEX COVER} instance is a yes instance then the created \textbf{SET COVER} instance is also a yes instance. 
  
  
  
  \medskip     
   
    \noindent \textbf{VERTEX COVER:} \\
INSTANCE: An undirected graph $G=(V,E)$ and integer $k$. \\
QUESTION: Does there exist a vertex cover $N$ of size $ \leq k$?
     i.e. $N \subseteq V$, s.t. for all $[i, j ] \in  E$, either $i \in N$ or $j \in N$? 
    

  \medskip     
  
\noindent \textbf{SET COVER:} \\
INSTANCE: A finite set $X$ of elements, a collection of $n$ subsets $S_i \subseteq X$, such that every element of $X$ belongs to at least one subset $S_i$, and an integer $m$. 
\\
QUESTION: Does there exist      a collection $C$ of at most $m$ of these subsets, such that the members of $C$ cover all elements of $X$? 
i.e. $\bigcup_{S \in C} S = X$. 
\\[1.1ex]
\textit{Example:} The following \SetCov instance: $X=\{1,2,3,4,5\}, S_1=\{1,2,3\}, S_2=\{3,4\}$, $S_3=\{1,2,5\}, S_4=\{4,5\}$ and $m=2$, is a yes instance, because there exists a collection $C$ with two subsets that cover all elements of $X$: $C=\{S_1, S_4\}$.
    

\end{exercise}

\solution{The reduction from \textbf{VERTEX COVER} to \textbf{SET COVER}:\\
  Let the finite set $X$ of elements be E, the Set of Edges. Let $n$ be $|V|$, the number of vertices in G. For every vertice u $\in$ V let $S_u$ be a subset of $X$ containing all the adjacent edges of u. Let $m=k$.
  This transformation is clearly done in polynomial time.
  The proof for $\Rightarrow$ direction, i.e., if a \textbf{VERTEX COVER} instance is a yes instance then the created \textbf{SET COVER} is also a yes instance:
  \begin {enumerate}
  \item{Assume \textbf{VERTEX COVER} is a yes instance, i.e., there exists a vertex cover N of size $\leq$ k. That means N $\subseteq$ V, s.t. for all [i,j] $\in$ E, either $i \in N$ or $j \in N$ }
  \item{We construct our solution for \textbf{SET COVER}: For every vertex u $\in$ N add the corresponding set $S_u$ to $C$.}
  \item{By our assumption in (a), only k vertices are in N and thus by (b), we can only have k sets in $C$. Since we stated in our reduction that $m=k$, this constraint is fulfilled.}
  \item{We now need to proof that every element in $X$ is in a subset $S_u$ which is in C.} 
  \item{Therefore let $x \in X$ be arbitrary.}
  \item{x by our reduction must be an edge $\in$ E}
  \item{By our reduction, (a) and (b), every edge is in one of the subsets $S_u$ which is in C. We know there is a set cover N (a). Every vertice u $\in$ N has a corresponding set $S_u$ in which all
    adjacent edges of u are in. Because for all [i,j] $\in$ E either $i \in N$ or $j \in N $ we conversly know that all vertices in N (and thus all subsets in C by (b) ) contain all edges adjacently.}
  \item{Thus all elements in of $X$ are in a subset $S_u$ which is in C, and we have a yes instance of set cover.}  
  \end {enumerate}
}



\begin{exercise}
  \label{ex:Scheduling}
  
  Provide a reduction from a simplified \textbf{EMPLOYEE SCHEDULING}  problem to \textbf{SAT}. 
  
 
  \medskip     
  
  
\noindent \textbf{EMPLOYEE SCHEDULING:}     

\noindent
INSTANCE:
\begin{itemize}
\item {Number of employees: $n$.}
\item {Set $A = \{D,A,N,-\}$ where $D =\ $"day shift", $A =\ $"afternoon shift", $N =\ $"night shift", and $- =\ $"day-off". }
\item {$w$: length of schedule. Suppose that $w=7$.}
\item {Temporal requirements: The requirement matrix $R$ $(3\times 7)$, where each element $r_{i,j}$ of the requirement matrix $R$ shows the required number of employees for shift $i$ during day $j$.}




\item {Constraints: 

\begin{itemize}
\item{Sequences of shifts not permitted to be assigned to employees. Suppose that these sequences are not permitted: $(N\ D)$, $(A\ D)$, and  $(N\ A)$. For example, the sequence $(N\ D)$ (Night Day) indicates that after working in the night shift, it is not allowed to work the next day in the day shift. }

\item{Maximum and minimum length of blocks of workdays: 
${MAXW}$, ${MINW}$. A Work block is a sequence consisting only from the working shifts (without day-off in beetwen). Suppose that $MAXW=5$ and $MINW=2$.}
\end{itemize}}
\end{itemize}

\noindent
QUESTION: Find a schedule (assignment of shifts to employees) that satisfies the requirement matrix, and the two given constraints. 

\medskip
\noindent
A schedule is represented by an $n\times w$ matrix $S\in A^{nw}$. Each element $s_{i,j}$ of matrix $S$ corresponds to one shift. Element $s_{i,j}$ shows on which shift employee $i$ works during day $j$ or whether the employee has day-off.

 
\medskip 
\noindent
EXAMPLE: Suppose that we are given an instance of \textbf{EMPLOYEE SCHEDULING} with $n=9$ and the following requirement matrix:

\[
R_{3,7} =
 \begin{pmatrix}
  2 & 2 & 2 & 2 & 2 & 2 & 2 \\
  2 & 2 & 2 & 2 & 2 & 2 & 2 \\
  2 & 2 & 2 & 2 & 2 & 2 & 2
  
 \end{pmatrix}
\]


\noindent In Table~\ref{typicalschedule} an employee schedule is presented that satisfies the requirement matrix, and the two given constraints in the problem definition. This schedule describes explicitly the working schedule of 9 employees during one week. The first employee works from Monday until Friday in a day shift (D) and during Saturday and Sunday has days-off. The second employee has days-off on Monday and Tuesday and works in a day shift during the rest of the week. Further, the last employee works from Monday until Wednesday in a night shift (N), on Thursday and Friday has days-off, and on Saturday and Sunday works in the day shift. Each row of this table represents the weekly schedule of one employee.      

\begin{table}[ht]
\renewcommand{\arraystretch}{1.2}
\caption{A typical week schedule for 9 employees }\label{typicalschedule}
\begin{center}
\begin{tabular}{||c|c|c|c|c|c|c|c||}  \hline
Emp./day & Mon & Tue & Wed & Thu & Fri & Sat &Sun \\ \hline \hline
1 & D & D & D & D & D & - & - \\ \hline 2 & - & - & D & D & D & D & D \\
\hline 3 & - & - & - & N & N & N & N \\ \hline 4 & D & D & - & - & A & A & A \\
\hline 5 & A & A & A & A & - & - & - \\ \hline 6 & N & N & N & N & N & - & - \\
\hline 7 & - & - & A & A & A & A & A \\ \hline 8 & A & A & - & - & - & N & N \\
\hline 9 & N & N & N & - & - & D & D \\ \hline
\end{tabular}
\end{center}
\end{table} 
 
 
 
\end{exercise}



%\solution{ Your solution here.}


\begin{exercise}

  \label{ex:Scheduling2}

  Give a proof sketch of the correctness of your reduction in the previous exercise. Does this reduction imply that \textbf{EMPLOYEE SCHEDULING} is an $\NP$-complete problem? Argue your answer.  
   
\end{exercise}


%\solution{ Your solution here.}



\begin{exercise}
  \label{ex:CO-NP}
Fomally prove that logical entailment in propositional logic is $co-\NP$-complete. 
 
\medskip   
\noindent 
\textbf{ENTAILMENT ( $\models$):} \\
\noindent 
INSTANCE: Propositional logic sentences $\alpha$ and $\beta$. 

\noindent 
QUESTION: Does the sentence $\alpha$ entails the sentence $\beta$  ($\alpha \models \beta$)? 

\medskip

\noindent $\alpha \models \beta$ if and only if, in every truth assignment in which $\alpha$ is true, $\beta$ is also true.    

  
  
\end{exercise}
%\solution{ Your solution here.}

\newpage

\begin{exercise}
  Consider the following problem:

  \begin{center}
    \fbox{
      \begin{minipage}[c]{.95\linewidth}
        \textbf{SELECT-3RD}

\medskip

INSTANCE: An integer $n$, and a list $L=(n_1,n_2,\ldots,n_m)$ of
integers, where no integer occurs twice in $L$ (i.e. $L$ represents a
set).\\
QUESTION: Is $n$ the third biggest element in $L$?
       \end{minipage}
    }
  \end{center}


  \medskip Provide a logarithmic space algorithm for
  solving~\textbf{SELECT-3RD} (use pseudo-code notation). Argue why it
  uses only logarithmic space.
\end{exercise}
\noindent \textbf{Hint.} Each element in $L$ can be addressed using a
pointer. Each pointer requires only logarithmic space.


\solution{
We have \textbf{SELECT-3RD} $\in$ L because we can traverse the list and use only five pointers, namely $first$, $second$, $third$, $tmp$ and $tmp2$ to store the respective
position in the set. After traversing the list, $third$ is then compared with $n$.}
\begin{lstlisting}[label=haltingtodouble,caption=CompareThird,mathescape, numbers=left, numberstyle=\tiny]
Boolean $CompareThird$ (List l, Integer n) {
  Element first;
  Element second;
  Element third;
  for(Element e : List l){
    if(first == null || first < e){
       Element tmp = first;
       Element tmp2 = second;
       first = e;
       second = tmp;
       third = tmp2;
       continue;
    }
    if(second == null || second < e){
       Element tmp = second;
       second = e;
       third = tmp;
       continue;
    }
    if(third == null || third < e){
      third = e;
      continue;
    }
  }
  return third==n;
}
\end{lstlisting}
The elements in $L$ are adressed with a pointer and because each pointer requires only logarithmic space and we only used pointers \textbf{SELECT-3RD} $\in$ L. 
The input to this program is assumed to be stored in the read-only memory and thus its space is not counted to the algorithm.


\begin{exercise}
  \label{ex:turing}

  Design a Turing machine that increments by one a positive value represented by a string of 0s and 1s.

\end{exercise}

\solution{
Turing machine M for incrementing a value represented by a string of 0s and 1s. 
M assumes that the lsb is left and the msb is right.
M = (K,$\Sigma$, $\delta$, s) with $K=\{s, q\}$, $\Sigma=\{0,1,\blank, \ssym\}$ and a transition function $\delta$ as follows:\\

\begin{center}
    \begin{tabular}{| l  l | l |}
    \hline
    $p \in K$ & $\sigma \in \Sigma$ & $\delta(p,\sigma)$\\ \hline
    s         &       $ \ssym  $    &  $ (s,\ssym,\rightarrow) $               \\ \hline
    s         &        0            &  $(\halt,1,\stay)$                 \\ \hline
    s         &        1            &  $(s,1,\rightarrow)$                 \\ \hline
    s         &       $ \blank $    &  $(q,1,\leftarrow)$                 \\ \hline
    q         &        1            &  $(q,0,\leftarrow)$                 \\ \hline
    q         &        $\ssym$      &  $(\halt,\ssym,\stay$)                 \\ \hline
    \hline
    \end{tabular}
\end{center}

}




\end{document}


